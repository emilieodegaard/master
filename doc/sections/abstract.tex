\chapter{Abstract}

The multiple network poroelasticity equations (MPET) describes mechanical deformation and fluid flow in porous media and can be used to understand various biological processes in a physiological setting. Modeling transportation of fluid within the brain is essential to discover the underlying mechanisms that are currently being investigated concerning various neurodegenerative diseases such as Alzheimer's disease. Mathematical modeling is considered to be more accessible and less expensive than performing advanced medical tests and experiments; however numerical simulations are still prone to error, making it essential to be able to control and minimize it. Physiological frameworks often include complex geometries which may produce complex error distributions. A posteriori error estimation presents a framework to measure and control the error in specific regions of the computational domain. This thesis presents the derivation of a posteriori error estimates for MPET with two interacting fluid networks, extending the analysis from one fluid network. Numerical experiments corroborate the theoretical results. The presented a posteriori error estimates can be extended to the MPET model with an arbitrary number of networks, which is demonstrated with a computational experiment using four networks on a brain mesh with physiologically inspired parameters.

