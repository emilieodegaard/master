\chapter{Discussion and conclusions}
\label{chap:discussion}
This thesis has studied residual-based a posteriori error estimation for the two-network poroelasticity model (i.e.~Barenblatt-Biot), with the main contribution presented in Theorem \ref{theorem}. This result gives the derivation of the a posteriori error estimates for the quasi-static Barenblatt-Biot model. The Barenblatt-Biot model is the simplest form for the generalized equations of poroelasticity (MPET) consisting of more than one network. From an application point of view, MPET models have been used for some time in geomechanics to model complex strata such as highly fissured reservoirs. Such structures are characterized by multiple fluid networks having distinct permeabilities, and porosities \cite{bai,Barenblatt1960,Barenblatt1963,Aifantis1982,Aifantis1984}. Recently, the flexibility offered by the MPET equations in modeling multiple permeable and porous networks has attracted the attention of communities working at the intersection of clinical application, applied mathematics, and biomedical engineering. In this context the simplest systems typically model an organ, e.g.~the brain, using four distinct fluid networks: arterial blood, capillary blood, venous blood, and an interstitial fluid or, in the brain, a combination of interstitial and cerebrospinal fluid \cite{vardakis}. 
\\
\\
Despite the advantages offered by the MPET model, such as accounting for several interacting fluid networks, the application of numerical methods within a complex tissue, such as the brain, faces additional challenges. Such challenges include multiple loading modes, compliant mechanical response, and regional variations in mechanical parameters, among others \cite{goriely}. Moreover, uncertainties in data acquisition can further obfuscate patient-specific simulations based on errors in parameter estimation such as medical imaging techniques. These practical concerns can lead to spatial and temporal errors in numerical simulations used to assist clinicians in patient diagnosis, or in computational models designed to test prominent clinical conjectures such as the glymphatic hypothesis \cite{iliff}. Thus, it is essential to control and potentially minimize the spatial and temporal errors.  
\\
\\
To improve a numerical solution, one must refine the spatial mesh in addition to the temporal interval of interest. In practice, a posteriori error estimates are often employed to refine in both space and time intelligently; one seeks to refine only in areas where the error estimators are large relative to the spatial and temporal discretization levels. Such a strategy can enhance the performance of numerical solvers as spatial refinement increases the size of the linear system to be solved, while time refinement increases the number of solution steps needed to deduce numerical results over the temporal interval. In order to explore the efficacy of the a posteriori estimates, numerical experiments, using the method of manufactured solutions, are put forth in section \ref{section:num_mpet2_default}. Due to the clinical motivation, and future applications for the work, the method of manufactured solutions is also employed on a mesh of a parenchymal slice of the mouse brain in section \ref{test_mpet4} using four fluid networks.  The mechanical parameters selected in the mouse brain test-case model correspond to those referenced in the context of clinical application \cite{vardakis,lee}. The boundary conditions considered, however, are purely clamped conditions; such boundary conditions are not physiological but are straightforward to implement. Nevertheless, this type of test offers some insight into the behavior of the estimators on geometries relevant to the application area. Clinical boundary conditions are more complex and require additional information, such as the production rate of cerebrospinal fluid in the ventricles \cite{guo,vardakis}, and are outside the scope of the current work.
%We have derived a posteriori error estimates 
%The main contribution of this thesis is given in Theorem (?) and is the derivation of the a posteriori estimates for the quasi-static Biot-Barenblatt model.  %The derivation is motivated by a similar approach for Biot's equation having its foundations in %
%based on 
%the work of Ern and Meunier \cite{meunier}. 
\\
\\
A posteriori error analysis for Biot's equation has been discussed in \cite{meunier, riedlbeck}; an a priori analysis can be found in \cite{murad1, murad2, murad3}. For the time-independent case, Nordbotten et al. have derived an a posteriori error estimator \cite{nordbotten}. This thesis has derived a posteriori error estimates for the quasi-static two-network case, i.e. the Barenblatt-Biot model. The derivation of our result, Theorem \ref{theorem}, is motivated by the work of Ern and Meunier \cite{meunier} for the quasi-static Biot equation. The primary differences between our result, and that of Ern and Meunier \cite{meunier}, are the addition of a second mass balance equation, and corresponding transfer terms, in extending to the two-network model. The extension is facilitated by in the analysis by augmenting the Sobolev norm, defined on the pressure space, to a norm including the effect of the transfer terms; this new norm is denoted by $\hat{d}$. The form of the arguments of Theorem \ref{theorem} suggest that extension to the case of four fluid networks is straightforward; as such, the Biot-Barenblatt model is the primary extension of concern, and we assign the full extension to the general multi-network case to future work. The remainder of this section offers additional detail for each fundamental component of the work; section \ref{section:a_posteriori} describes the a posteriori estimates and the extension to the two-network case, section \ref{section:mpet_numerical_res} the numerical tests for the two-network and the four network case, section \ref{section:conclusion} offers concluding thoughts, and section \ref{section:further_work} outlines some limitations of the model and future work. 

%We have performed numerical experiments to evaluate the estimates, where we performed a simple experiment and a physiologically inspired experiment on the unit square with two networks. In addition, we presented a purely computational experiment with a four-network MPET model using a brain mesh and physiologically relevant parameters. For this experiment, the analysis of the a posteriori error estimates was not presented as it is a simple extension of the two-network model. 

%A posteriori error analysis for MPET with one network has been covered in \cite{meunier, riedlbeck} and an a priori error analysis can be found in \cite{murad1, murad2, murad3}. Nordbotten et al. have derived an a posteriori error estimator for the static two-network case \cite{nordbotten}. This thesis has derived a posteriori error estimates for the quasi-static two-network case, i.e. the Barenblatt-Biot model. 

\section{A posteriori error estimates} \label{section:a_posteriori}
The a posteriori error estimates for the two-network MPET model is an extension from the one-network MPET model which has been derived in \cite{meunier}. The in-detail analysis and derivation for the one-network model can be found in section \ref{section:error_biot} where in particular the proof structure in section \ref{biot_proof_struc} forms the basis for the extension to the two-network model. The primary differences between our result, and that of Ern and Meunier \cite{meunier}, are the addition of a second mass balance equation, and corresponding transfer terms, in extending to the two-network model. Assuming non-interacting fluid networks will only differ from the one-network model in the added mass balance equation. The derivation of the a posteriori error estimate for MPET with non-interacting fluid networks will thus follow the same arguments as in \cite{meunier} and has been outlined in section \ref{bb:case1_res_err}. In the case of interacting fluid networks, the extension is facilitated by in the analysis by augmenting the Sobolev norm to a new norm denoted by $\hat{d}$ defined in equation \eqref{d_hat_norm}. This norm is defined on the pressure space to include the effect of the transfer terms. The main result containing the upper bound and lower bound on the error for the two-network MPET model is presented in Theorem \ref{theorem}. This result depends on the stability of the continuous problem. In order to arrive at the upper bound, all the equations are summed to get a total error on the left-hand side and the residual-terms on the right-hand side, see equation \eqref{bb:add_forms}. Similarly, for the lower bound, see equations \eqref{mpet2_low_bd1}-\eqref{mpet2_low_bd2} we arrive at one bound for the first equation \eqref{math_model:mpet2_u} and one bound for the pressure terms in \eqref{math_model:mpet2_p1}-\eqref{math_model:mpet2_p2} in section \ref{section:mpet}. Thus, the extension of the a posteriori error estimate to an arbitrary number of networks will not need any additional analysis but should be able to follow the same arguments as presented in this work. 
\\
\\
The a posteriori error estimates for the two-network MPET model was derived in chapter \ref{chap:error} where the main results were presented in Proposition \ref{prop} and Theorem \ref{theorem}. We derived four different estimators for the upper bound of the error, denoted as $\eta_1$, $\eta_2$, $\eta_3$ and $\eta_4$. These estimators provide different indications on how the error behaves in time and space, where $\eta_1$, $\eta_2$ and $\eta_3$ are space estimators and $\eta_4$ is a time estimator. The estimator $\eta_1$ is associated with the spatial residual of the displacement $u$, and $\eta_2$ is associated with the spatial residual of the pressure $p$. $\eta_3$ is defined as the time incremental version of $\eta_1$ and will predict how time may affect the residual of the displacement under space refinement. In other words, $\eta_3$ heuristically measures the change in the displacement-related spatial residual in time. Thus, if $\eta_1$ is small compared to $\eta_3$ indicates further refinement in time. Conversely, if $\eta_3$ is small compared to $\eta_1$, indicates further refinement in space. The magnitude of these two estimators provides the necessary information on \textit{how} to refine. $\eta_4$ is a time estimator associated with the pressure, predicting how time will affect the pressure-solution.
\\
\\
In order to derive the estimates, we assumed that the exact solution of the unknowns was smooth in time and space. This may be a limitation in some applications, e.g. fracture reservoirs in geomechanical engineering which may include solutions with discontinuities. However, in biomedical applications, we do not encounter this specific type of problem. The a posteriori error estimators constructed for the MPET model will naturally depend on the model parameters. That is, if a parameter changes, the estimators associated with that parameters also changes. This was demonstrated in the experiment executed with physiologically relevant parameters in section  \ref{test_bb} where we observed a proportional relationship between the estimators and their associated model parameters.


\section{Numerical results} \label{section:mpet_numerical_res}
This section presents the main results from the numerical experiments outlined in chapter \ref{chap:experiments}, which included the evaluation of the a posteriori error estimates for the two-network and four-network poroelasticity model. 

\subsection{MPET: 2 networks}
For the two-network poroelasticity model with interacting fluid networks, using default parameters all set to 1 yields optimal convergence rates as expected from the a priori error estimates presented in section  \ref{section:a_priori}. That is, a $H^1$-rate and a $L^2$-rate of second order for the displacement and the pressures, respectively under space refinement, and of first order under time refinement, cf. table \ref{tab:bb_default_transfer_space_error} and table \ref{tab:bb_default_transfer_time_error}. The a posteriori error estimates converge optimally under time refinement, cf. table \ref{tab:bb_default_transfer_time_est}. However, under space refinement, the estimator predicting the change in the displacement-related spatial residual in time converges at an order lower than expected from the analysis in \cite{meunier}, cf. table \ref{tab:bb_default_transfer_space_est}. The potential of the a posteriori error estimates is demonstrated in cf. figures \ref{fig:bb_default_eta1}, \ref{fig:bb_default_eta2}, \ref{fig:bb_default_eta3} and \ref{fig:bb_default_eta4} as the error magnitudes indicate where the error is concentrated for the residual related to the displacement and the pressure under time and space refinement. 
\\
\\
Applying physiologically inspired parameters on the two-network poroelasticity model with interacting fluid networks yields an increasing convergence rate for the displacement and oscillating behavior for the pressure under spatial refinement, cf. table \ref{tab:bb_bio_space_error}. This computational behavior is known as "locking" and occurs when the displacement is underestimated due to large variations in the size of the model parameters. This is a well-known problem when implementing a two-field discretization, and a proposed solution is to implement e.g. the total pressure formulation presented in \cite{lee, lee2018} and the stabilization technique suggested in \cite{rodrigo}. The a posteriori estimates derived in this work will detect if locking occurs, which is demonstrated by a proportional relationship between the estimates and their associated model parameter, cf. table \ref{tab:bb_bio_space_est} and \ref{tab:bb_bio_time_est}.

\subsection{MPET: 4 networks}
The motivation behind using the MPET equations is to simulate fluid transportation in the brain. In light of this, it is important to be able to control the error on complex geometries such as a brain mesh. Thus, we presented a posteriori error magnitudes for a four-network MPET model on a mouse brain mesh in section \ref{test_mpet4}. The experiment implements the a posteriori error estimates derived for the two network poroelasticity model in chapter \ref{chap:error} extended to four networks. This shows that the analytic results do not depend on the network number. The experiment used simplified boundary conditions, which are not considered physiological. Thus, we are unable to view how the estimators may detect high feature variation in the various parts of the brain. 
\\
\\
Applying physiologically inspired parameters on the four-network poroelasticity model on a brain mesh yields similar results to the two-network model. This was expected, as the only difference between the two experiments was the two additional pressure-terms. We only performed one uniform space refinement, as this is a computationally expensive procedure on a complex geometry, which ratifies the goal of using the a posteriori error estimates for adaptive refinement in this application framework.


\section{Conclusions} \label{section:conclusion}
A posteriori error estimates provide insight on how to intelligently refine in time and space. The goal of applying the a posteriori error estimates is to be able to control the error and refine in the areas where it is needed. Uniform refinement is computationally expensive, whereas adaptive refinement offers a way to decrease the error while maintaining a minimal number of grid points. The a posteriori error estimates are a prerequisite to performing adaptive refinement as they will predict how the error behaves on each mesh cell. The main contribution in this thesis is the derivation of residual-based a posteriori estimates for the quasi-static Barenblatt-Biot model, which is the simplest form for the generalized equations of poroelasticity (MPET) consisting of more than one network. The potential of the estimators has been demonstrated using the technique of manufactured solutions and a poroelasticity benchmark. The estimators yield upper bounds on the error, which included space, time and data estimators. Numerical experiments corroborate the theoretical results. The presented a posteriori error estimates can be extended to the MPET model with an arbitrary number of networks, which was demonstrated with a computational experiment using four networks on a brain mesh.

\section{Further work} \label{section:further_work}
We have derived a posteriori error estimates for the multiple network poroelasticity model with two networks; however, analysis for the general MPET equations with an arbitrary number of networks is desirable.
\\
\\
We encountered the issue of locking when using standard mixed finite element discretization in a nearly incompressible case. An optimal discretization technique is a prerequisite to ensure an optimal a posteriori error estimate, and we suggest implementing an extension to mixed methods and derive the estimators subsequently to ensure robustness. 
\\
\\
The present work can be extended to the use of time-dependent meshes and adaptive simulations using the a posteriori error estimators to evaluate performance in terms of accuracy, precision, and efficiency. In addition, we suggest applying different error estimation techniques, e.g. goal-oriented, hierarchical, H(div)-lifting. 
\\
\\
We also suggest implementing a physiologically relevant test case including boundary conditions, model parameters, and exact solutions, as we believe the a posteriori error estimates may detect high feature contrasts in the various parts of the brain; this is valuable for future biomedical simulations. 